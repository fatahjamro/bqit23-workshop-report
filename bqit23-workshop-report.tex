\documentclass[a4paper, 12pt]{scrartcl}

\title{Report: BQIT23 -Bristol Quantum Information Technology Workshop 2023}
\subtitle{Quantum Engineering Technology (QET)Labs, University of Bristol, UK}

\author{Abdul Fatah Jamro}
\date{24-28 April 2023 }

\begin{document}


\maketitle
  10th annual conference BQIT-23 organised by Quantum Engineering Technology
  Labs (QETlabs) Universityof Bristol, Uk.

\section{Introduction}
  BQIT-23 was 10th annual quantum information technology conference organised 
  by QETLabs University of Bristol, Uk. Researchers participated and presented 
  their research work in the field of quantum technology. There were around 250
  young researchers as participants and around 40 experts speakers from academia and industry.
  Most of the speakers and participants were involved in the advanced research in quantum photonics. 
  The conference also covered topics like 

\begin{itemize}
  \item Quantum integrated photonic chips.
  \item Quantum foundation.
  \item Quantum dots.
  \item Satellite quantum communication (Photonics).
  \item Interferometry.
  \item Dark matter and gravity waves.
  \item Quantum computing.
\end{itemize}

Lab tour was also arranged where we got opportunity to see quantum integrated photonic processor,
photonic quantum refregirators, control and readout devices.

\section{Speakers}
  Over all workshop covered the broad scope of quantum physics but many speakers were related to Photonics field. Huge research and development is carried-out
  in the field of quantum photonics due to its applications in the field of electronnic and
  communication. Some of the talks also highlighted the possibility of computation obeying the laws of quantum physics.
  \subsection{Few focused presentations}
  \textbf{Gavin Morley} (University of Warwick, Uk) presented his talk on the topic \textit{Nitrogen-vacancy centres in 
  diamond for sensitive magnetometry and (eventually) a test of quantum gravity.} \\
  \textbf{Giulia Rubino} (University of Bristol) presented on \textit{A new operational approach
  to measure work in coherent quantum systems.}\\
  \textbf{Sarah Malik} (UCL) presented \textit{Quantum computing for particle physics.} She spoted light on the
  importance of quantum computing. Quantum computing looks poised to be one of the most transformative technologies 
  of the 21st century, with the potential to play a disruptive role in both science and society. 
  The current intermediate-scale quantum devices provide excellent test beds for performing proof 
  of principle studies on using quantum computers to tackle the most challenging problems in 
  particle physics.\\
  \textbf{Cristian Bonato} (Heriot-Watt University) discussed \textit{Silicon Carbide spin-based quantum devices for quantum networking} Single 
  optically-active spin defects and impurities have been used in many of the leading implementations of quantum networking. 
  Most world-leading experiments, such as the first loophole-free Bell test and the first demonstration of 
  teleportation in a three-node network have been implemented using the NV centre in diamond. Diamond has, however, 
  several drawbacks in terms of cost, commercial availability and lack of established fabrication recipes.\\
  \textbf{Thalia D} (University of Southampton) topic \textit{Advanced Silicon Nitride Integration for CMOS Photonic Circuits} 
  Silicon photonics has accelerated the deployment of complementary metal-oxide semiconductor (CMOS) compatible photonic integrated 
  circuits (PICs) based on the silicon-on-insulator (SOI) platform.\\
  \textbf{Stasja Stanisic} (Senior quantum Engineer at Phase-craft) topic \textit{Towards Practical Quantum Advantage}
  Phasecraft is the quantum algorithms company based in Bristol and London UK. Phasecraft partners with Google, IBM, and Rigetti.\\
  \textbf{Stasja} discussed that the noisy intermediate-scale quantum era has picked up speed in recent years, with 
  a varied range of companies now offering regular access to devices of diverse qubit numbers, gate errors, 
  and hardware connectivity. The promise of practical quantum advantage is becoming more achievable. Phasecraft performed 
  experiment on the GoogleSycamore devices to solve for the ground state of the FermiHubbard model using the 
  \textbf{variational quantum eigensolver}. Phasecraft is the team of more than 30 scientists, mathematicians,
   engineers and researchers.\\
  \textbf{Edmund Harbord} (University of Bristol) topic \textit{“More light, more light!” How can we design high brightness, high yield manufacturable 
  sources of single and entangled photons using quantum dots?} Quantum dots (QDs) – nanoscale inclusions of one semiconductor embedded in another 
  are ideal sources of single and entangled photons, with internal quantum efficiencies of ~100\%. 


\end{document}